In den Projekten im Rahmen des Moduls \enquote{Embedded Systems} wurde eine Weiterleitung von Paketen zwischen mehreren Projektteams simuliert. Die Basis bildete ein Master-Slave-Protokollstack (MMCP-Protokoll), der auf NUCLEO-F401RE Mikrocontrollerboards implementiert wurde. Zur Kommunikation wurden die Mikrocontroller-Boards auf einem sogenannten Baseboard zusammengeschaltet, wodurch die verschiedenen Teams in einer Kette über RS-232 mit dem Master kommunizieren konnten. Weitere Details zum MMCP-Protokoll und zum Baseboard finden sich im Skript des Moduls. \\
Die Paketweiterleitung wurde von jedem Team individuell als eine Applikation auf dem MMCP-Protkoll nach dem SA/RT-Entwurfsprinzip realisiert. In diesem Projekt wurde zur Visualisierung der Paketübergabe eine Paketdrohne auf einer LED-Matrix simuliert, welche ihren Weg durch ein zufällig generiertes Labyrinth zum Übergabepunkt des Nachbarteams findet. \\
In diesem Projektbericht wird zunächst das vollständige SA/RT-Modell vorgestellt, dann die Hardware erläutert und anschließend auf die Software für die LED-Ansteuerung und das Labyrinth eingegangen. Abschließend werden Tests des Projektes mit den Nachbarteams beschrieben.